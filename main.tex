\documentclass[a4paper, 12pt, twppages]{article}
\usepackage[utf8]{inputenc}
\usepackage{graphicx}
\usepackage{eso-pic}
\usepackage{fancyhdr}
\usepackage{pdfpages}
\usepackage{hyperref}
\usepackage{cleveref}

\begin{document}
\pagebreak

\thispagestyle{fancy}


{\vspace{2cm}~}
{\vspace{2cm}}

{\noindent\large Project Paper}
\vspace{1cm}

{\noindent\huge\textbf{Using a Random Forest for Static Analysis of Android Malware}}

\bigskip

{\noindent\LARGE Group 2}


\bigskip
{\vspace{2cm}}
{\noindent\large {\bf Supervisor:} Nils Löhndorf}

\bigskip

%% WRITE THE DATAE OF SUBMISSION
{\noindent\large {\bf Date of Submission:} 26. January 2018}

\bigskip\bigskip\bigskip\bigskip\bigskip\bigskip

{\em\noindent Department of Information Systems and Operations, Vienna University of
Economics and Business, Welthandelsplatz 1, 1020 Vienna, Austria
}


\pagestyle{empty}
\pagebreak
\tableofcontents
\pagebreak
\listoftables
\pagebreak


\begin{abstract}
Malware is a constant threat to users of the Android operating system. In regions where Google Play is not used as application distributor, malware increasingly exceeds benignware. We try to solve this problem by using Machine Learning algorithms to automatically detect either benign or malicious intents of Android applications. This way, users could more easily be notified of previously unknown malware. We use the Drebin dataset \cite{drebin} four our classifier. It includes around 120.000 benignware and 6.000 malware samples. We train various Machine Learning algorithms on the Drebin dataset and show that the Random Forest algorithm surpasses the performance of the SVM used in \cite{drebin}, reaching a true positive rate of 95 \% and a false positive rate of 0.6 \%. We also were provided with a newer dataset with proper class balance by IKARUS Security Software GmbH, but could not finish our analysis on it due to time constraints.
\end{abstract}

\pagebreak

\section{Introduction}
\emph{In cooperation with Aaron Kaplan (from CERT.at) and Hans Christoph
	Steiner (from the Guardian Project)}

Our project heavily involves a conference paper from 2014 titled
\href{https://www.tu-braunschweig.de/Medien-DB/sec/pubs/2014-ndss.pdf}{DREBIN:
	Effective and Explainable Detection of Android Malware in Your Pocket}.
We focus on recreating a similar situation, but adjusted for todays
state-of-the-art software and test the effectiveness of the papers
program DREBIN with modern mal- and benignware. To achieve a meaningful
result, we have to manage the following objectives:

\subsection{Feature Engineering}\label{feature-engineering}

In order to supply the learning algorithms with data, we have to extract
the specific properties and features of the APKs. The main obstacles in
this part of the project are:

\begin{enumerate}
	\def\labelenumi{\arabic{enumi}.}
	\item
	Finding out what we are looking for (What makes an APK malware?)
	\item
	Knowing where to look for
\end{enumerate}

As already mentioned, the first step in this section is the reproduction
of the program that extracts the features from an APK used in the DREBIN
paper. Afterwards, the goal is to improve the existing feature
extraction and look for new features. Interesting properties of an apk
may be:

\begin{itemize}
	\item
	Requested permissions
	\item
	URLs used for communication
	\item
	Method calls
	\item
	Activity names\\
	etc\ldots{}
\end{itemize}

An APK is basically a .zip file, which contains many other files, some
are mandatory and some are optional. Luckily, we already have a list of
APK features, which are known from the DREBIN paper to give sufficient
information for the learning algorithms. The main sources of information
are the Mainfest.xml (containing metadata about the app) and the
classes.dex files. The .dex file contains compiled code, which must be
decompiled in order to analyze it.

The result of this is a
\href{https://github.com/33onethird/feature-extraction}{Feature-Extractor}

\subsection{Classification}\label{classification}

The objective is to optimize a binary classification of the Android app
samples. We will try a range of machine learning algorithms to
discriminate between malware (positive class) and benignware (negative
class). The performance of the algorithms will be compared using a
ROC-curve. This curve plots the true-positive rate of the test dataset
against the false-positive rate.

The techniques used in the Drebin paper achieved a true-positive rate of
94 \% at a false-positive rate of 1 \%. We will try to achieve these
numbers with our copy using the original dataset and possibly surpass
that result. How could we surpass a true-positive rate of 94 \%. There
are three potential improvements of current techniques:

\begin{enumerate}
	\def\labelenumi{\arabic{enumi}.}
	\item
	Gather more features and improve existing features (as described in
	the feature engineering section)
	\item
	Use deep learning. The papers approaches scarcely made use of deep
	learning techniques. As our features are numerical, we can use neural
	networks for the task.
	\item
	More training data. The dataset used for malware classification in the
	2014 paper contained approximately 130000 labelled applications. This
	was considered the biggest dataset for this use at the time.
	Thankfully, our partners at Cert.at and the GuardianProject could
	provide us with considerably larger quantity of additional new
	applications to recognise the evolution of malware.
\end{enumerate}

The new training data especially will be important to test the
effectiveness of DREBIN with modern software and document the changes in
the detection rate.

\href{https://github.com/33onethird/malware-test}{Here is the repository
	about how the malware detection is installed and how to use it}

\section{Malware Detection}
Insert text


\section{Feature Extraction}
\label{sec:fe}

Clearly, analyzing every file manually does not scale very well. In
order to detect malware automatically, the application files must
be analyzed by a program. We are interested in information such as
DNS names used for communication, potentially malicious method calls
(e.g. sending premium SMS), usage of known malware libraries and similar
behavior. Generally speaking, there are two fundamentally different
approaches to this topic: Static and dynamic software analysis.\cite{staticdynamic}

\subsection{Static Analysis}

Using static analysis, the decompiled code of a program is inspected.
The required information is extracted out of several thousand lines
of code by using techniques, such as regular expressions or search
lists. These methods require an understanding of the format and syntax
of the analyzed code and the parsing program must be tailored to one
specific use case. Therefore, the writing of the parsing program is
preceded by a manual inspection of the target code, where the programmer
defines the goals of the extraction. Finding correct and valid extraction
patterns is essential for the static analysis. This project focuses
on application of static analysis methods.\\
A main advantage of this approach is the inspection of the entire
code, instead of only executed code. The provided information does
not change and can be inspected before the actual execution of the
program. It is also possible to find the exact line of code, which
is responsible for a certain behavior of an application. Therefore,
the workflow can be reconstructed and offers potentially a deep understanding
of the software.\\
However, the static analysis approach also has its limitations. The
most problematic one is the obfuscation of code. It is possible to
hardcode an encrypted domain name in the software, which is decrypted
if the program wants to communicate with this domain name. The static
analysis program can only extract the encrypted name, which is most
likely useless to the analyst. Another way of obfuscating code is
downloading malicious programs in a seemingly benign application.
Again, the static analysis cannot analyze the malicious program, because
it is downloaded on runtime. Additionally, it is possible to generate
false positive and false negative results, because of imprecise extraction
patterns or wrong assumptions of the analyst. 

\subsection{Dynamic Analysis }

The usage of dynamic software analysis always involves some kind of
sandbox for the software sample. By installing and running the software
in a controlled environment, its behavior is logged and can be analyzed.
This method allows a more practically oriented approach, because individual
branches of code are target of inspection, instead of the entire decompiled
code. It is also possible to see changes during runtime and observe
every variable. \\
The biggest advantage over the static analysis is certainly the possibility
to observe the program during runtime. Therefore, obfuscation techniques,
such as the encryption of domain names or the downloading of malicious
code can be negated. It is also no longer necessary to manually inspect
the decompiled application and write a program specifically tailored
to a specific use case. There are several tools, which automatically
analyze the program during runtime. These are universally applicable
and require little to none tailoring.\\
A drawback of this approach is the execution of individual software
branches. Most likely, these branches depend on user input and it
is not always possible to test an application with every thinkable
combination of user input. Furthermore, it is possible for an application
to detect, whether it is executed on a real system or just a sandbox
and change its behavior accordingly. Therefore, the dynamic analysis
may deem a program as benign, which changed its behavior during the
analysis and generate false positive or negatives. It is also important
to keep in mind that the tools used for the analysis may not detect
every activity or generate false logfiles. 



\section{Machine Learning}
This section will detail our approach to classify Android apps into malware and benignware. We utilize Machine Learning methods for this task. Specifically, we expand upon the Machine Learning systems introduced in \cite{drebin} and \cite{7917369}.

Obviously, we need training data for our classifier. We use two datasets: the Drebin dataset used in \cite{drebin} and another dataset that we were provided with by IKARUS Security Software GmbH\footnote{\url{https://www.ikarussecurity.com}} (henceforth referred to as \emph{Ikarus dataset}). The Drebin dataset contains 123,453 benignware samples and 5,560 malware samples. The Ikarus dataset contains 29,148 benignware samples and 29,627 malware samples. Additionaly, it supplies 29,774 adware samples and 29,957 unlabeled samples, which we ignore in our analysis.

When comparing the two datasets, one sees that the Drebin dataset displays a significant class imbalance of around 1:22. On the other hand, the Drebin dataset supplies more total observations. It is also important to note that the Drebin dataset contains Android apps published before 2014, while the Ikarus dataset contains Android apps published in 2017. It is clear that the Ikarus dataset is generally more suitable to the classification task. We use both datasets to reproduce the results of \cite{drebin} and compare the two approaches.

The dataset consists solely of binary features (see \Cref{sec:fe}). The label is also binary (0: benignware, 1: malware). The Drebin dataset provides a 545,334-dimensional feature space, while the Ikarus dataset provides a 316,256-dimensional feature space.

The classification system detailed in \cite{drebin} utilizes the linear SVM algorithm. This poses an issue, since the linear SVM assumes real-valued inputs. The assumption holds in our case, but it does not utilize the binary nature of our features. Therefore, we additionally use three other algorithms which utilize categorical features and labels: \emph{Bernoulli Naive Bayes} \cite{Manning:2008:IIR:1394399}, \emph{Logistic Regression} \cite{10.2307/2983890} and \emph{Random Forests} \cite{598994}. Bernoulli Naive Bayes mainly serves as robust baseline.

We also implemented a neural network for classification. We used a \emph{multi-layer perceptron} (MLP) \cite{rosenblatt1958}, beause the local connectivity of \emph{convolutional neural networks} (CNN) \cite{Fukushima1980} and the memory capabilities of \emph{recurrent neural networks} (RNN) \cite{doi:10.1162/neco.1997.9.8.1735} were contrary to our assumptions. We figure that the MLP could improve the classifications due to two capabilities: learning low-dimensional representations of apps and detecting feature hierarchies. Due to lack of resources, we were not able to validate this assumption.
\section{Case Study}
\subsection{Experiment Setup}
\subsection{Results}
\section{Related Work}

The primary goal of this project was the reproduction of DREBIN, a
virus scanner for android created in 2014.\footnote{\cite{drebin}}
Its aim was to provide a lightweight method of detecting malware on
the Android platform using machine learning. This goal was realized
by an app, which analyzed and classified .apk files directly in the
phone within ten seconds. \\
The main contributions of the DREBIN projects were: 
\begin{itemize}
\item A highly effective detection of malware and consequently high true
positive (94\%) and low false positive (1\%) rates. This was achieved
via machine learning, avoiding manually crafted detection patterns. 
\item The results are transparent and explainable, because of the chosen
workflow. The classification provides an insight can be traced back
to the raw data. 
\item By applying linear time analysis, the classification process is very
efficient and can be executed directly on the smartphone. Furthermore,
the analysis of a large dataset may finish in a reasonable amount
of time.
\end{itemize}
They achieve these goals by conducting a broad static analysis on
the files of an application. Precisely, the manifest and disassembled
dex code are used to extract certain feature sets. The entire process
is based on strings and the feature files are also stored as such.
Features extracted from the manifest are: Hardware components (e.g.
camera, GPS), Requested permissions (e.g. SEND\_SMS), App components
(e.g. component names) and Filtered intents (e.g. acting on BOOT\_COMPLETED).
The disassembled code provides the following features: Restricted
API calls (e.g. calling a method without permission), Used permission
(e.g. permissions actually used in the code), Suspicious API calls
(e.g. executing external commands via Runtime.exec() ) and Network
addresses (e.g. URLS). 


\section{Discussion}
\subsection{Out of Scope}\label{out-of-scope}

\paragraph{Dynamic Analysis}\label{dynamic-analysis}

We did not work with dynamic analysis, because the performance on mobile
devices would decrease massively. We work exclusively with static
analysis

\paragraph{Obfuscation}\label{obfuscation}

If the malware is being obfuscated then the decompilation would face
problems, because countering obfuscation requires a whole new project
and was not further included in these methods

\paragraph{Entropy/Randomness}\label{entropyrandomness}

We tried to detect and measure randomness of strings inside the code,
but it turned out, that most urls would appear random and this would
sharply increase the false positive rate. This method would require a
list of known bad urls.

\paragraph{Semantic Analysis}\label{semantic-analysis}

This type of analysis is used to approximate concepts through structures
build from a large set of information. This is out of scope.

\subsection{Namesake}\label{namesake}

Our organization heavily revolves around the infamous
\href{https://www.tu-braunschweig.de/Medien-DB/sec/pubs/2014-ndss.pdf}{DREBIN}
paper, which shares the name with a beloved Leslie Nielsen character
from the filmseries ``The Naked Gun''. The movie lives off of slapstick
and puns, much like we do.
\section{Conclusio}
We used two datasets - the Drebin dataset and the Ikarus dataset, both consisting of malware and beningware samples - to train various Machine Learning algorithms to detect malware. We showed that our implementation of a Random Forest surpassed the results of the original paper \cite{drebin}. We note that using the newer Ikarus dataset will yield better performance on state-of-the-art malware. We further note that any practical Machine Learning-based malware detection application will have to perpetually train on newer malware and benignware samples in order to keep up with changing technologies. We want to thank our instructors and collaborators for their efforts.




\bibliographystyle{plain}
\bibliography{references}

\end{document}
