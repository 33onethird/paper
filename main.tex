\documentclass[a4paper, 12pt, twppages]{article}
\usepackage[utf8]{inputenc}
\usepackage{graphicx}
\usepackage{eso-pic}
\usepackage{fancyhdr}
\usepackage{pdfpages}
\usepackage{hyperref}
\usepackage{cleveref}

\begin{document}
\pagebreak

\thispagestyle{fancy}


{\vspace{2cm}~}
{\vspace{2cm}}

{\noindent\large Project Paper}
\vspace{1cm}

{\noindent\huge\textbf{Using a Random Forest for Static Analysis of Android Malware}}

\bigskip

{\noindent\LARGE Group 2}


\bigskip
{\vspace{2cm}}
{\noindent\large {\bf Supervisor:} Nils Löhndorf}

\bigskip

%% WRITE THE DATAE OF SUBMISSION
{\noindent\large {\bf Date of Submission:} 26. January 2018}

\bigskip\bigskip\bigskip\bigskip\bigskip\bigskip

{\em\noindent Department of Information Systems and Operations, Vienna University of
Economics and Business, Welthandelsplatz 1, 1020 Vienna, Austria
}


\pagestyle{empty}
\pagebreak
\tableofcontents
\pagebreak
\listoftables
\pagebreak


\begin{abstract}
Malware is a constant threat to users of the Android operating system. In regions where Google Play is not used as application distributor, malware increasingly exceeds benignware. We try to solve this problem by using Machine Learning algorithms to automatically detect either benign or malicious intents of Android applications. This way, users could more easily be notified of previously unknown malware. We use the Drebin dataset \cite{drebin} four our classifier. It includes around 120.000 benignware and 6.000 malware samples. We train various Machine Learning algorithms on the Drebin dataset and show that the Random Forest algorithm surpasses the performance of the SVM used in \cite{drebin}, reaching a true positive rate of 95 \% and a false positive rate of 0.6 \%. We also were provided with a newer dataset with proper class balance by IKARUS Security Software GmbH, but could not finish our analysis on it due to time constraints.
\end{abstract}

\pagebreak

\section{Introduction}
Insert introduction
\section{Malware Detection}
Insert text
\section{Feature Extraction}
insert
\section{Machine Learning}
\section{Case Study}
\subsection{Experiment Setup}
\subsection{Results}
\section{Related Work}

The primary goal of this project was the reproduction of DREBIN, a
virus scanner for android created in 2014.\footnote{\cite{drebin}}
Its aim was to provide a lightweight method of detecting malware on
the Android platform using machine learning. This goal was realized
by an app, which analyzed and classified .apk files directly in the
phone within ten seconds. \\
The main contributions of the DREBIN projects were: 
\begin{itemize}
\item A highly effective detection of malware and consequently high true
positive (94\%) and low false positive (1\%) rates. This was achieved
via machine learning, avoiding manually crafted detection patterns. 
\item The results are transparent and explainable, because of the chosen
workflow. The classification provides an insight can be traced back
to the raw data. 
\item By applying linear time analysis, the classification process is very
efficient and can be executed directly on the smartphone. Furthermore,
the analysis of a large dataset may finish in a reasonable amount
of time.
\end{itemize}
They achieve these goals by conducting a broad static analysis on
the files of an application. Precisely, the manifest and disassembled
dex code are used to extract certain feature sets. The entire process
is based on strings and the feature files are also stored as such.
Features extracted from the manifest are: Hardware components (e.g.
camera, GPS), Requested permissions (e.g. SEND\_SMS), App components
(e.g. component names) and Filtered intents (e.g. acting on BOOT\_COMPLETED).
The disassembled code provides the following features: Restricted
API calls (e.g. calling a method without permission), Used permission
(e.g. permissions actually used in the code), Suspicious API calls
(e.g. executing external commands via Runtime.exec() ) and Network
addresses (e.g. URLS). 


\section{Discussion}
\subsection{Out of Scope}\label{out-of-scope}

\paragraph{Dynamic Analysis}\label{dynamic-analysis}

We did not work with dynamic analysis, because the performance on mobile
devices would decrease massively. We work exclusively with static
analysis

\paragraph{Obfuscation}\label{obfuscation}

If the malware is being obfuscated then the decompilation would face
problems, because countering obfuscation requires a whole new project
and was not further included in these methods

\paragraph{Entropy/Randomness}\label{entropyrandomness}

We tried to detect and measure randomness of strings inside the code,
but it turned out, that most urls would appear random and this would
sharply increase the false positive rate. This method would require a
list of known bad urls.

\paragraph{Semantic Analysis}\label{semantic-analysis}

This type of analysis is used to approximate concepts through structures
build from a large set of information. This is out of scope.

\subsection{Namesake}\label{namesake}

Our organization heavily revolves around the infamous
\href{https://www.tu-braunschweig.de/Medien-DB/sec/pubs/2014-ndss.pdf}{DREBIN}
paper, which shares the name with a beloved Leslie Nielsen character
from the filmseries ``The Naked Gun''. The movie lives off of slapstick
and puns, much like we do.
\section{Conclusio}
We used two datasets - the Drebin dataset and the Ikarus dataset, both consisting of malware and beningware samples - to train various Machine Learning algorithms to detect malware. We showed that our implementation of a Random Forest surpassed the results of the original paper \cite{drebin}. We note that using the newer Ikarus dataset will yield better performance on state-of-the-art malware. We further note that any practical Machine Learning-based malware detection application will have to perpetually train on newer malware and benignware samples in order to keep up with changing technologies. We want to thank our instructors and collaborators for their efforts.




\bibliographystyle{plain}
\bibliography{references}

\end{document}
