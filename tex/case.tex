\section{Results}
This section details the results of our project. The resulting implementations are located at \url{https://github.com/33onethird}. A demo web page can be accessed at \url{http://svm-google.ai.wu.ac.at:8080/api/v1/submit}. We use the best performing algorithm for the demo. We implemented the Machine Learning algorithms using the Python library scikit-learn \cite{scikit-learn}.

\Cref{tab:resdreb} details the true positive rate and false positive rate of the Machine Learning algorithms we used in our analysis of the Drebin dataset. We trained the algorithms on $40,000$ samples of the Drebin dataset and evaluated it on another $10,000$ samples. Note that we neither exhaust the dataset nor perform an exhaustive search on the hyperparameter space due to limited resources. Nevertheless, we see a clear indication that the Random Forest algorithm outperforms the other methods. See \Cref{tab:rfdrebin,,tab:svmdrebin,,tab:lrdrebin} to inspect the hyperparameters.

Notice that we do not list results for the Ikarus dataset, as they training process is not yet completed.
\begin{table}[p]
	\centering
	\begin{tabular}{l r r}
		Model & TPR in \% & FPR in \%\\
		\hline
		Random Forest & 95 & 0.6\\
		SVM & 90 & 0.5\\
		Logistic Regression & 84 & 0.6\\
		Naive Bayes & 52 & 2.4\\
		\hline
	\end{tabular}
	\caption{Results of various algorithms on the Drebin dataset}
	\label{tab:resdreb}
\end{table}

\begin{table}[p]
	\centering
	\begin{tabular}{l r}
		Hyperparameter & Value \\
		\hline
		Number of estimators (trees) & $1,000$\\
		Criterion & Gini coefficient\\
		Class 0 (benignware) weight & $1$\\
		Class 1 (malware) weight & $2$\\
		\hline
	\end{tabular}
	\caption{Hyperparameters for the Random Forest used on the Drebin dataset}
	\label{tab:rfdrebin}
\end{table}

\begin{table}[p]
	\centering
	\begin{tabular}{l r}
		Hyperparameter & Value \\
		\hline
		$C$ & $1.5$\\
		Class 0 (benignware) weight & $1$\\
		Class 1 (malware) weight & $2$\\
		\hline
	\end{tabular}
	\caption{Hyperparameters for the SVM used on the Drebin dataset}
	\label{tab:svmdrebin}
\end{table}

\begin{table}[p]
	\centering
	\begin{tabular}{l r}
		Hyperparameter & Value \\
		\hline
		dual & no\\
		solver & saga\\
		$C$ & $1$\\
		Iterations & $10,000$\\
		Class 0 (benignware) weight & $1$\\
		Class 1 (malware) weight & $2$\\
		\hline
	\end{tabular}
	\caption{Hyperparameters for the Logistic Regression used on the Drebin dataset}
	\label{tab:lrdrebin}
\end{table}