\section{Introduction}
\emph{In cooperation with Aaron Kaplan (from CERT.at) and Hans Christoph
	Steiner (from the Guardian Project)}

Our project heavily involves a conference paper from 2014 titled
\href{https://www.tu-braunschweig.de/Medien-DB/sec/pubs/2014-ndss.pdf}{DREBIN:
	Effective and Explainable Detection of Android Malware in Your Pocket}.
We focus on recreating a similar situation, but adjusted for todays
state-of-the-art software and test the effectiveness of the papers
program DREBIN with modern mal- and benignware. To achieve a meaningful
result, we have to manage the following objectives:

\subsection{Feature Engineering}\label{feature-engineering}

In order to supply the learning algorithms with data, we have to extract
the specific properties and features of the APKs. The main obstacles in
this part of the project are:

\begin{enumerate}
	\def\labelenumi{\arabic{enumi}.}
	\item
	Finding out what we are looking for (What makes an APK malware?)
	\item
	Knowing where to look for
\end{enumerate}

As already mentioned, the first step in this section is the reproduction
of the program that extracts the features from an APK used in the DREBIN
paper. Afterwards, the goal is to improve the existing feature
extraction and look for new features. Interesting properties of an apk
may be:

\begin{itemize}
	\item
	Requested permissions
	\item
	URLs used for communication
	\item
	Method calls
	\item
	Activity names\\
	etc\ldots{}
\end{itemize}

An APK is basically a .zip file, which contains many other files, some
are mandatory and some are optional. Luckily, we already have a list of
APK features, which are known from the DREBIN paper to give sufficient
information for the learning algorithms. The main sources of information
are the Mainfest.xml (containing metadata about the app) and the
classes.dex files. The .dex file contains compiled code, which must be
decompiled in order to analyze it.

The result of this is a
\href{https://github.com/33onethird/feature-extraction}{Feature-Extractor}

\subsection{Classification}\label{classification}

The objective is to optimize a binary classification of the Android app
samples. We will try a range of machine learning algorithms to
discriminate between malware (positive class) and benignware (negative
class). The performance of the algorithms will be compared using a
ROC-curve. This curve plots the true-positive rate of the test dataset
against the false-positive rate.

The techniques used in the Drebin paper achieved a true-positive rate of
94 \% at a false-positive rate of 1 \%. We will try to achieve these
numbers with our copy using the original dataset and possibly surpass
that result. How could we surpass a true-positive rate of 94 \%. There
are three potential improvements of current techniques:

\begin{enumerate}
	\def\labelenumi{\arabic{enumi}.}
	\item
	Gather more features and improve existing features (as described in
	the feature engineering section)
	\item
	Use deep learning. The papers approaches scarcely made use of deep
	learning techniques. As our features are numerical, we can use neural
	networks for the task.
	\item
	More training data. The dataset used for malware classification in the
	2014 paper contained approximately 130000 labelled applications. This
	was considered the biggest dataset for this use at the time.
	Thankfully, our partners at Cert.at and the GuardianProject could
	provide us with considerably larger quantity of additional new
	applications to recognise the evolution of malware.
\end{enumerate}

The new training data especially will be important to test the
effectiveness of DREBIN with modern software and document the changes in
the detection rate.

\href{https://github.com/33onethird/malware-test}{Here is the repository
	about how the malware detection is installed and how to use it}
