\section{Related Work}

The primary goal of this project was the reproduction of DREBIN, a
virus scanner for android created in 2014.\footnote{\cite{drebin}}
Its aim was to provide a lightweight method of detecting malware on
the Android platform using machine learning. This goal was realized
by an app, which analyzed and classified .apk files directly in the
phone within ten seconds. \\
The main contributions of the DREBIN projects were: 
\begin{itemize}
\item A highly effective detection of malware and consequently high true
positive (94\%) and low false positive (1\%) rates. This was achieved
via machine learning, avoiding manually crafted detection patterns. 
\item The results are transparent and explainable, because of the chosen
workflow. The classification provides an insight can be traced back
to the raw data. 
\item By applying linear time analysis, the classification process is very
efficient and can be executed directly on the smartphone. Furthermore,
the analysis of a large dataset may finish in a reasonable amount
of time.
\end{itemize}
They achieve these goals by conducting a broad static analysis on
the files of an application. Precisely, the manifest and disassembled
dex code are used to extract certain feature sets. The entire process
is based on strings and the feature files are also stored as such.
Features extracted from the manifest are: Hardware components (e.g.
camera, GPS), Requested permissions (e.g. SEND\_SMS), App components
(e.g. component names) and Filtered intents (e.g. acting on BOOT\_COMPLETED).
The disassembled code provides the following features: Restricted
API calls (e.g. calling a method without permission), Used permission
(e.g. permissions actually used in the code), Suspicious API calls
(e.g. executing external commands via Runtime.exec() ) and Network
addresses (e.g. URLS). 

