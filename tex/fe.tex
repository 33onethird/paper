

\section{Feature Extraction}

Clearly, analyzing every file manually does not scale very well. In
order to detect malware automatically, the application files must
be analyzed by a program. We are interested in information such as
DNS names used for communication, potentially malicious method calls
(e.g. sending premium SMS), usage of known malware libraries and similar
behavior. Generally speaking, there are two fundamentally different
approaches to this topic: Static and dynamic software analysis.\cite{staticdynamic}

\subsection{Static Analysis}

Using static analysis, the decompiled code of a program is inspected.
The required information is extracted out of several thousand lines
of code by using techniques, such as regular expressions or search
lists. These methods require an understanding of the format and syntax
of the analyzed code and the parsing program must be tailored to one
specific use case. Therefore, the writing of the parsing program is
preceded by a manual inspection of the target code, where the programmer
defines the goals of the extraction. Finding correct and valid extraction
patterns is essential for the static analysis. This project focuses
on application of static analysis methods.\\
A main advantage of this approach is the inspection of the entire
code, instead of only executed code. The provided information does
not change and can be inspected before the actual execution of the
program. It is also possible to find the exact line of code, which
is responsible for a certain behavior of an application. Therefore,
the workflow can be reconstructed and offers potentially a deep understanding
of the software.\\
However, the static analysis approach also has its limitations. The
most problematic one is the obfuscation of code. It is possible to
hardcode an encrypted domain name in the software, which is decrypted
if the program wants to communicate with this domain name. The static
analysis program can only extract the encrypted name, which is most
likely useless to the analyst. Another way of obfuscating code is
downloading malicious programs in a seemingly benign application.
Again, the static analysis cannot analyze the malicious program, because
it is downloaded on runtime. Additionally, it is possible to generate
false positive and false negative results, because of imprecise extraction
patterns or wrong assumptions of the analyst. 

\subsection{Dynamic Analysis }

The usage of dynamic software analysis always involves some kind of
sandbox for the software sample. By installing and running the software
in a controlled environment, its behavior is logged and can be analyzed.
This method allows a more practically oriented approach, because individual
branches of code are target of inspection, instead of the entire decompiled
code. It is also possible to see changes during runtime and observe
every variable. \\
The biggest advantage over the static analysis is certainly the possibility
to observe the program during runtime. Therefore, obfuscation techniques,
such as the encryption of domain names or the downloading of malicious
code can be negated. It is also no longer necessary to manually inspect
the decompiled application and write a program specifically tailored
to a specific use case. There are several tools, which automatically
analyze the program during runtime. These are universally applicable
and require little to none tailoring.\\
A drawback of this approach is the execution of individual software
branches. Most likely, these branches depend on user input and it
is not always possible to test an application with every thinkable
combination of user input. Furthermore, it is possible for an application
to detect, whether it is executed on a real system or just a sandbox
and change its behavior accordingly. Therefore, the dynamic analysis
may deem a program as benign, which changed its behavior during the
analysis and generate false positive or negatives. It is also important
to keep in mind that the tools used for the analysis may not detect
every activity or generate false logfiles. 


